% arara: xelatex
%% arara: xelatex


\documentclass[14pt,xcolor=dvipsnames]{beamer}


% !TEX root = om_metrics_14.tex

%\usepackage{epsdice} % dice 1-6 for probability :)

% \usepackage[absolute,overlay]{textpos}

% \usefonttheme[onlymath]{serif}

\usefonttheme{professionalfonts}
% by default beamer changes math fonts for better visibility for projection
% this professionalfonst theme removes this behavior


\usepackage[orientation=portrait,size=custom,width=25.4,height=19.05]{beamerposter}




%25,4 см 19,05 см размеры слайда в powerpoint

\usetheme{metropolis}
\metroset{
  %progressbar=none,
  numbering=none,
  subsectionpage=progressbar,
  block=fill
}

%\usecolortheme{seahorse}

\usepackage{fontspec}
\usepackage{polyglossia}
\setmainlanguage{english}


% \usepackage{fontawesome5} % removed [fixed]
\setmainfont[Ligatures=TeX]{Myriad Pro}
% \setsansfont{Myriad Pro}




% why do we need \newfontfamily:
% http://tex.stackexchange.com/questions/91507/
\newfontfamily{\cyrillicfonttt}{Myriad Pro}
\newfontfamily{\cyrillicfont}{Myriad Pro}
%\newfontfamily{\cyrillicfontbs}{Myriad Pro}
\newfontfamily{\cyrillicfontsf}{Myriad Pro}


% https://tex.stackexchange.com/questions/175860/why-does-unicode-math-break-the-kerning-of-accents-in-combination-with-amssymb
% "You shouldn't be using amssymb together with unicode-math"
\usepackage{amsmath}
\usepackage{amsthm} % amssymb 


% https://tex.stackexchange.com/questions/483722/
% \usepackage[MnSymbol]{mathspec}  % Includes amsmath.
% \usepackage{mathspec}  % Includes amsmath.
% \setmathsfont(Digits,Latin,Greek,Symbols)[Numbers={Lining,Proportional}]{Latin Modern Math}
% mathspec must be loaded earlier than amsmath



%\usepackage{bm}

% \usepackage{fdsymbol} % \nperp

% \usepackage{unicode-math} % \symbf
% \setmathfont{Latin Modern Math}



\usepackage{centernot}

\usepackage{graphicx}

\usepackage{wrapfig}
% \usepackage{animate} % animations :)
% \usepackage{tikz}
%\usetikzlibrary{shapes.geometric,patterns,positioning,matrix,calc,arrows,shapes,fit,decorations,decorations.pathmorphing}
% \usepackage{pifont}
\usepackage{comment}
\usepackage[font=small,labelfont=bf]{caption}
\captionsetup[figure]{labelformat=empty}
% \includecomment{techno}



%Расположение

\setbeamersize{text margin left=15 mm,text margin right=5mm} 
\setlength{\leftmargini}{38 pt}

%\usepackage{showframe}
%\usepackage{enumitem}
% \setlist{leftmargin=5.5mm}


%Цвета от дирекции

\definecolor{dirblack}{RGB}{58, 58, 58}
\definecolor{dirwhite}{RGB}{245, 245, 245}
\definecolor{dirred}{RGB}{149, 55, 53}
\definecolor{dirblue}{RGB}{0, 90, 171}
\definecolor{dirorange}{RGB}{235, 143, 76}
\definecolor{dirlightblue}{RGB}{75, 172, 198}
\definecolor{dirgreen}{RGB}{155, 187, 89}
\definecolor{dircomment}{RGB}{128, 100, 162}

\setbeamercolor{title separator}{bg=dirlightblue!50, fg=dirblue}

%Цвета блоков

% Голубой блок!
\setbeamercolor{block title}{bg=dirblue!30,fg=dirblack}
\setbeamercolor{block title example}{bg=dirlightblue!50,fg=dirblack}
\setbeamercolor{block body example}{bg=dirlightblue!20,fg=dirblack}

\AtBeginEnvironment{exampleblock}{\setbeamercolor{itemize item}{fg=dirblack}}
%\setbeamertemplate{blocks}[rounded][shadow]

% Набор команд для удобства верстки

% Набор команд для структуризации

%\newcommand{\quest}{\faQuestionCircleO}
%\faPencilSquareO \faPuzzlePiece \faQuestionCircleO  \faIcon*[regular]{file} {\textcolor{dirblue}
%\newcommand{\quest}{\textcolor{dirblue}{\boxed{\textbf{?}}}
%\newcommand{\task}{\faIcon{tasks}}
%\newcommand{\exmpl}{\faPuzzlePiece}
%\newcommand{\dfn}{\faIcon{pen-square}}
%\newcommand{\quest}{\textcolor{dirblue}{\faQuestionCircle[regular]}}
%\newcommand{\acc}[1]{\textcolor{dirred}{#1}}
%\newcommand{\accm}[1]{\textcolor{dirred}{#1}}
%\newcommand{\acct}[1]{\textcolor{dirblue}{#1}}
%\newcommand{\acctm}[1]{\textcolor{dirblue}{#1}}
%\newcommand{\accex}[1]{\textcolor{dirblack}{\bf #1}}
%\newcommand{\accexm}[1]{\textcolor{dirblack}{ \mathbf{#1}}}
%\newcommand{\acclp}[1]{\textcolor{dirorange}{\it #1}}
\newcommand{\todo}[1]{\textcolor{dircomment}{\bf #1}}
%\newcommand{\graylink}[1]{{\fontsize{11}{12}\selectfont \textcolor{gray}{#1}}}
%\newcommand{\figcaption}[1]{{\fontsize{18}{20}\selectfont #1}}


\newcommand{\videotitle}[1]{
    {\fontsize{33}{30}\selectfont \textcolor{dirblue}{\textbf{#1}} }

    %\todo{название видеофрагмента}
}

\newcommand{\lecturetitle}[1]{
  {\fontsize{33}{30}\selectfont \textcolor{dirblue}{\textbf{#1}} }

    %\todo{название лекции}
}





%\newcommand{\spcbig}{\vspace{-10 pt}}
%\newcommand{\spcsmall}{\vspace{-5 pt}}

%\usepackage{listings}
%\lstset{
%xleftmargin=0 pt,
%  basicstyle=\small, 
%  language=Python,
  %tabsize = 2,
%  backgroundcolor=\color{mc!20!white}
%}



%\newcommand{\mypart}[1]{\begin{frame}[standout]{\huge #1}\end{frame}}

\setbeamercolor{background canvas}{bg=}

% frame title setup
\setbeamercolor{frametitle}{bg=,fg=dirblue}
\setbeamertemplate{frametitle}[default][left]

\addtobeamertemplate{frametitle}{\hspace*{0.1 cm}}{\vspace*{0.25cm}}


%Шрифты
\setbeamerfont{frametitle}{family=\rmfamily,series=\bfseries,size={\fontsize{33}{30}}}
\setbeamerfont{framesubtitle}{family=\rmfamily,series=\bfseries,size={\fontsize{26}{20}}}


% удобнее знать номер слайда, чтобы вносить правки!  

\setbeamercolor{footline}{fg=dircomment}
\setbeamerfont{footline}{series=\bfseries, size={\fontsize{12}{14}}}
%\setbeamertemplate{footline}[page number]


\defbeamertemplate{footline}{custom footline}
{%
  \hspace*{\fill}%
  \usebeamercolor[fg]{page number in head/foot}%
  \usebeamerfont{page number in head/foot}%
  page: \insertpagenumber\,/\,\insertpresentationendpage%
  \hspace{20pt}%
  slide: \insertframenumber\,/\,\inserttotalframenumber%
  %\hspace*{\fill}
  \vskip2pt%
}
%\setbeamertemplate{footline}[custom footline]

\usepackage{physics}


% tikz block

\usepackage{pgfplots}
\pgfplotsset{compat=newest}

\usepackage{tikz}
\usetikzlibrary{calc}
\usetikzlibrary{quotes,angles}
\usetikzlibrary{arrows}
\usetikzlibrary{arrows.meta}
\usetikzlibrary{positioning,intersections,decorations.markings}
\usetikzlibrary{patterns}
\usepackage{tikzsymbols}

\usepackage{tkz-euclide} 
\usepackage{tikzducks}

%\tikzset{>=latex}


\tikzset{cross/.style={cross out, draw=black, minimum size=2*(#1-\pgflinewidth), inner sep=0pt, outer sep=0pt},
%default radius will be 1pt. 
cross/.default={5pt}}

\colorlet{veca}{red}
\colorlet{vecb}{blue}
\colorlet{vecc}{olive}


\newcommand{\grid}{\draw[color=gray,step=1.0,dotted] (-2.1,-2.1) grid (9.6,6.1)}

% end tikz block

\newcommand{\R}{\mathbb{R}}
\newcommand{\Rot}{\mathrm{R}}
\newcommand{\HH}{\mathrm{H}}
\newcommand{\Id}{\mathrm{I}}
\newcommand{\RR}{\mathbb{R}}
\newcommand{\ZZ}{\mathbb{Z}}
\newcommand{\la}{\lambda}
\let\P\relax
\newcommand{\P}{\mathbb{P}}
\newcommand{\E}{\mathbb{E}}

\newcommand{\cN}{\mathcal{N}}
\newcommand{\cF}{\mathcal{F}}

\newcommand{\dN}{\mathcal{N}}

\newcommand{\qL}{q_{\text{left}}}
\newcommand{\qR}{q_{\text{right}}}



\newcommand{\ba}{\mathbf{a}}
\newcommand{\be}{\mathbf{e}}
\newcommand{\bb}{\mathbf{b}}
\newcommand{\bc}{\mathbf{c}}
\newcommand{\bd}{\mathbf{d}}
\newcommand{\bx}{\mathbf{x}}
\newcommand{\bff}{\mathbf{f}} % \bf is already def
\newcommand{\bv}{\mathbf{v}}
\newcommand{\bzero}{\mathbf{0}}


\DeclareMathOperator{\Var}{Var}
\DeclareMathOperator{\sVar}{sVar}
\DeclareMathOperator{\Cov}{Cov}
\DeclareMathOperator{\sCov}{sCov}
\DeclareMathOperator{\sCorr}{sCorr}
\DeclareMathOperator{\Corr}{Corr}

\DeclareMathOperator{\plim}{plim}


\newcommand{\graylink}[1]{{\fontsize{11}{12}\selectfont \textcolor{gray}{#1}}}
\newcommand{\figcaption}[1]{{\fontsize{18}{20}\selectfont #1}}




\newcommand{\knightduck}{
\raisebox{-10pt}{%
\tikz[scale=0.6]{%
	\duck[helmet]
	\fill[gray!60!black] (0.5831,0.5821) -- (0.8112,0.6468) -- (0.7528,0.7563) .. controls (0.7367,0.7865) and (0.8427,0.8231) .. (0.8497,0.7897) -- (0.8749,0.6705) -- (0.9316,0.5183) -- (0.9744,0.4102) .. controls (0.9865,0.3796) and (0.8848,0.3549) .. (0.8784,0.3872) -- (0.8553,0.5051) -- (0.6311,0.4379) -- (0.6541,0.3774) .. controls (0.6619,0.3569) and (0.6051,0.3422) .. (0.5908,0.3589) .. controls (0.5908,0.3589) and (0.5068,0.4303) .. (0.4875,0.4797) .. controls (0.4692,0.5265) and (0.4818,0.6305) .. (0.4818,0.6305) .. controls (0.4809,0.6553) and (0.5387,0.6862) .. (0.5484,0.6634) -- cycle;
	\fill[gray] (0.8749,0.6705) -- (1.6884,0.9465) -- (1.9041,0.8933) -- (1.7728,0.7282) -- (0.9316,0.5183) -- cycle;%
}}}

\newcommand{\formalduck}{
\raisebox{-10pt}{%
\tikz[scale=0.55]{%
	\duck[tophat,bowtie,tshirt,jacket=gray]%
}}}


\newcommand{\informalduck}{
\raisebox{-10pt}{%
\tikz[scale=0.6]{%
	\duck[crazyhair]%
}}}



\newcommand{\harlequinduck}{
\raisebox{-10pt}{%
\tikz[scale=0.6]{%
    \duck[harlequin=blue,
    niuqelrah=red]%
}}}

\begin{document}


\begin{frame} % название лекции


\lecturetitle{\textbf{Itô integral}}

\end{frame}


% !TEX root = ../coursera_sc_02.tex

\begin{frame} % название фрагмента

\videotitle{Itô integral}

\end{frame}


\begin{frame}{Itô integral: short plan}

  \begin{itemize}[<+->]
    \item Intuitive definition.
    \item Examples.
  \end{itemize}

\end{frame}


\begin{frame}{Itô integral}

  \begin{block}{Definition\informalduck}
    The Itô integral $I_t = \int_0^t A_u \, dW_u$ is defined as the \alert{total net cash flow} of a strategy if we treat 
    $W_u$ as the price of the asset and $A_u$ as the quantity of the asset at time moment $u$. 
\end{block}

\pause 
$(W_t)$ is a poor model for the price, but the intuition is ok:
\pause
\[
\text{Net cash flow} = \int_0^t \text{Quantity}_u \, d\text{Price}_u  
\]
\end{frame}

\begin{frame}
  \frametitle{Simple deterministic example}
  
  Let's calculate $\int_2^8 7 \, du$\knightduck

  \onslide<2->{Transaction 1. At time $u=2$ \alert{we buy 7 units}. The price is $u=2$.}

  \onslide<3->{Cash flow: $-7 \cdot 2$.}

  \onslide<4->{Transaction 2. At time $u=8$ \alert{we sell 7 units}. The price is $u=8$.}

  \onslide<5->{Cash flow: $7 \cdot 8$.}

  \onslide<6->{$\int_2^8 7 \, du = -14 + 56 = 42$.}

\end{frame}


\begin{frame}
  \frametitle{Example with Wiener process}
  
  Let's calculate $\int_2^8 5 \, dW_u$\knightduck

  \onslide<2->{Transaction 1. At time $u=2$ \alert{we buy 5 units}. The price is $W_2$.}

  \onslide<3->{Cash flow: $-5 \cdot W_2$.}

  \onslide<4->{Transaction 2. At time $u=8$ \alert{we sell 5 units}. The price is $W_8$.}

  \onslide<5->{Cash flow: $5 \cdot W_8$.}

  \onslide<6->{$\int_2^8 5 \, dW_u = -5W_2 + 5W_8$.}

\end{frame}

\begin{frame}
  \frametitle{More gentlemen's agreement\formalduck}

  \[
  I_t = \int_0^t (\text{something}_u) \, dW_u:
  \]

  $t$ — the upper limit of integration;

  $u$ — time variable with range from $0$ to $t$;

\end{frame}


\begin{frame}
  \frametitle{Why old integration formulas are wrong?}
  \[
  \int_0^t W_u \, dW_u = \harlequinduck?  
  \]
  \pause
  
  Why not $\frac{1}{2}W_t^2 - \frac{1}{2} W_0^2$?
  \pause

  The guessed value is non-negative, $\frac{1}{2}W_t^2 \geq 0$! 

  \pause 
  If you buy and sell Wiener process you can have negative cash flow!  
\end{frame}


\begin{frame}
  \frametitle{Small table}
  In most cases Itô integral \alert{can not} be computed explicitely. 
  \pause
  \[
    \int_0^t 1 \, dW_u = W_t
  \]
  \pause
  \[
    \int_0^t W_t \, dW_u = \frac{W_t^2 - t}{2}
  \]
  \pause
  \[
    \int_0^t \exp(aW_u -\frac{1}{2} a^2 u) \, dW_u = \frac{1}{a}\left(\exp(aW_t - \frac{1}{2}a^2 t) - 1\right) 
  \]

\end{frame}



\begin{frame}{Itô integral: summary}
    
  \begin{itemize}[<+->]
      \item \alert{Total net cash flow} of a strategy.
      \item \alert{Rarely} can be computed \alert{explicitely}.
      \item \alert{Old rules} of integration \alert{do not apply}.
  \end{itemize}
    
\end{frame}
  

% !TEX root = ../coursera_sc_02.tex


% !TEX root = ../coursera_sc_02.tex

\begin{frame} % название фрагмента

    \videotitle{Itô integral properties}
    
    \end{frame}
    
    
    \begin{frame}{Itô integral properties: short plan}
    
      \begin{itemize}[<+->]
        \item \alert{Common properties} with Riemann integral.
        \item \alert{Zero expected value}.
        \item \alert{Itô isometry}.
      \end{itemize}
    
    \end{frame}


\begin{frame}
        \frametitle{Common properties}

        \[
        \int_a^b X_u \, dW_u  + \int_b^c X_u \, dW_u =  \int_a^c X_u \, dW_u    
        \]
        \pause 
        \[
        \int_a^a X_u \, dW_u = 0    
        \]
        \pause
        \[
        \int_0^t c X_u \, dW_u = c\int_0^t X_u \, dW_u    
        \]
    
\end{frame}


\begin{frame}
    \frametitle{Zero expected value}

    Intuition: we buy and sell Wiener process, hence, expected net cash flow should be zero.

    \pause
\begin{block}{Informal theorem\informalduck}
For any reasonable process $(X_t)$ measurable with respect to the natural filtration $(\cF_t)$ of the process $(W_t)$
\[
\E\left(\int_0^t X_u \, dW_u \right)  = 0.   
\]
\end{block}

\end{frame}

    
\begin{frame}
    \frametitle{Itô isometry}

\begin{block}{Informal theorem\informalduck}
        For any reasonable process $(X_t)$ measurable with respect to the natural filtration $(\cF_t)$ of the process $(W_t)$
        \[
        \Var\left(\int_0^t X_u \, dW_u \right)  = \int_0^t \E(X_u^2) \, du.   
        \]
\end{block}
            
\end{frame}


\begin{frame}
    \frametitle{Exercise}

    Find $\E(I_t)$ and $\Var(I_t)$ for $I_t = \int_0^t W_u^2 \, dW_u$\knightduck

    \begin{flalign*}
        \onslide<2->{\E(I_t) &=}\onslide<3->{0;&}  
    \end{flalign*} 

    \begin{flalign*}
        \onslide<4->{\Var(I_t) & =}\onslide<5->{\int_0^t \E(W_u^4) \, du = }
        \onslide<6->{\int_0^t 3 u^2  \, du =  }\onslide<7->{ t^3.&}
    \end{flalign*} 
    
\end{frame}


\begin{frame}{Itô properties: summary}
        
\begin{itemize}[<+->]
          \item \alert{Zero} expected value.
          \item Using \alert{Itô isometry} one may calculate variance.
          \item \alert{Some common properties} with Riemann integral.
\end{itemize}
        
\end{frame}
      


% !TEX root = ../coursera_sc_02.tex


\begin{frame} % название фрагмента

    \videotitle{Itô process}
    
    \end{frame}
    
    
    \begin{frame}{Itô process: short plan}
    
      \begin{itemize}[<+->]
        \item \alert{Definition} of an Itô process.
        \item When Itô process is a \alert{martingale}?
        \item \alert{Short} and \alert{full} form notation.
      \end{itemize}
    
    \end{frame}

\begin{frame}
    \frametitle{Itô process}

    \begin{block}{Definition\formalduck}
        Stochastic process $(Y_t, t\geq 0)$ is called \alert{Itô process} if it can be written in the form 
        \[
        Y_t = Y_0 + \int_0^t A_u \, dW_u + \int_0^t B_u \, du,
        \]
        where $Y_0$ is a constant.
    \end{block}

    \pause
    A wide class of continuous stochastic processes that behave \alert{locally} like a Wiener process with drift. 

\end{frame}


\begin{frame}
    \frametitle{Itô integral is a martingale}

    \begin{block}{Informal theorem\informalduck}
        Itô process $(Y_t)$ is a martingale if and only if it has only Itô integral in the representation
        \[
        Y_t = Y_0 + \int_0^t A_u \, dW_u.
        \]
    \end{block}

    \pause
    The best guess of a future value of an Itô integral is its current value:
    \pause
    \[
    \E(Y_t \mid \cF_s) = Y_s \text{ for } s \leq t.
    \]

\end{frame}


\begin{frame}
    \frametitle{Expected value of Itô process}
    \begin{block}{Informal theorem\informalduck}
        For any reasonable process $(B_t)$ 
        \[
        \E\left(\int_0^t B_u \, du\right) = \int_0^t \E(B_u) \, du.
        \]
    \end{block}

    \pause
    
    \begin{block}{Informal theorem\informalduck}
        If $(Y_t)$ is an Itô process with $Y_t = Y_0 + \int_0^t A_u \, dW_u + \int_0^t B_u \, du$, 
        then
        \[
        \E(Y_t) = Y_0 + \int_0^t \E(B_u) \, du.
        \]
    \end{block}    

\end{frame}


\begin{frame}
    \frametitle{Short form notation}

    Full form:
    \[
        Y_t = Y_0 + \int_0^t A_u \, dW_u + \int_0^t B_u \, du.
    \]
    \pause
    Short form:
    \[
        dY_t = A_t \, dW_t + B_t \, dt.
    \]
    \pause
    $dW_t$ and $dY_t$ have \alert{no meaning}!

\end{frame}

\begin{frame}
    \frametitle{Short form in simulations}
    We need to simulate a path of 
    \[
        Y_t = Y_0 + \int_0^t A_u \, dW_u + \int_0^t B_u \, du.        
    \]
    \pause
    In short form:
    \[
    dY_t = A_t \, dW_t + B_t \, dt. 
    \]
    \pause
    In simulations:
    \[
    Y_{t+\Delta} - Y_t \approx A_t \cdot (W_{t + \Delta} - W_t) + B_t \cdot \Delta,     
    \]
    where $W_{t + \Delta} - W_t \sim \cN(0; \Delta)$.
\end{frame}


\begin{frame}
    \frametitle{Short form: examples}
    \begin{flalign*}
        \onslide<1->{dY_t = W_t^4 dW_t & }\onslide<2->{ \quad Y_t = Y_0 + \int_0^t W_u^4 \, dW_u.&}
    \end{flalign*} 
    \begin{flalign*}
        \onslide<3->{dY_t = \cos(W_t) dt & }\onslide<4->{ \quad Y_t = Y_0 + \int_0^t \cos(W_u) \, du.&}
    \end{flalign*} 

    \onslide<5->{
    \begin{block}{Informal theorem\informalduck}
        Itô process $(Y_t)$ is a martingale if and only if 
        \[
        dY_t = A_t \, dW_t.
        \]
    \end{block}
    }
\end{frame}


\begin{frame}{Itô process: summary}
        
\begin{itemize}[<+->]
          \item A \alert{sum three terms}: constant, Itô integral and Riemann integral.
          \item Will be a martingale \alert{without Riemann integral}.
          \item Often written using \alert{short form} with $dt$ and $dW_t$.
\end{itemize}
        
\end{frame}
      


\end{document}
