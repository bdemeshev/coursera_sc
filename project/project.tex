% arara: xelatex
%% arara: xelatex

\documentclass[12pt]{article}

\usepackage{tikz} % картинки в tikz
\usepackage{microtype} % свешивание пунктуации

\usepackage{array} % для столбцов фиксированной ширины

\usepackage{indentfirst} % отступ в первом параграфе

\usepackage{sectsty} % для центрирования названий частей
\allsectionsfont{\centering}

\usepackage{amsmath, amssymb, amsthm} % куча стандартных математических плюшек

\usepackage{amsfonts}

\usepackage{comment}

\usepackage[top=2cm, left=1.2cm, right=1.2cm, bottom=2cm]{geometry} % размер текста на странице

\usepackage{lastpage} % чтобы узнать номер последней страницы

\usepackage{enumitem} % дополнительные плюшки для списков
%  например \begin{enumerate}[resume] позволяет продолжить нумерацию в новом списке
\usepackage{caption}


\usepackage{hyperref} % гиперссылки

\usepackage{multicol} % текст в несколько столбцов


\usepackage{fancyhdr} % весёлые колонтитулы
\pagestyle{fancy}
\lhead{Introduction to Stochastic Calculus}
\rhead{Project}

\cfoot{}
\rfoot{}
% \rfoot{\thepage/3}
\renewcommand{\headrulewidth}{0.4pt}
\renewcommand{\footrulewidth}{0.4pt}



% \usepackage{todonotes} % для вставки в документ заметок о том, что осталось сделать
% \todo{Здесь надо коэффициенты исправить}
% \missingfigure{Здесь будет Последний день Помпеи}
% \listoftodos - печатает все поставленные \todo'шки


% более красивые таблицы
\usepackage{booktabs}
% заповеди из документации:
% 1. Не используйте вертикальные линии
% 2. Не используйте двойные линии
% 3. Единицы измерения - в шапку таблицы
% 4. Не сокращайте .1 вместо 0.1
% 5. Повторяющееся значение повторяйте, а не говорите "то же"



\usepackage{fontspec}
\usepackage{polyglossia}

\setmainlanguage{english}
\setotherlanguages{english}

% download "Linux Libertine" fonts:
% http://www.linuxlibertine.org/index.php?id=91&L=1
\setmainfont{Linux Libertine O} % or Helvetica, Arial, Cambria
% why do we need \newfontfamily:
% http://tex.stackexchange.com/questions/91507/
\newfontfamily{\cyrillicfonttt}{Linux Libertine O}

%\AddEnumerateCounter{\asbuk}{\russian@alph}{щ} % для списков с русскими буквами
%\setlist[enumerate, 2]{label=\asbuk*),ref=\asbuk*}

%% эконометрические сокращения
\let\P\relax
\DeclareMathOperator{\P}{\mathbb{P}}
\DeclareMathOperator{\Cov}{\mathbb{C}ov}
\DeclareMathOperator{\Corr}{\mathbb{C}orr}
\DeclareMathOperator{\Var}{\mathbb{V}ar}
\DeclareMathOperator{\E}{\mathbb{E}}
\DeclareMathOperator{\tr}{trace}
\newcommand \hb{\hat{\beta}}
\newcommand \hs{\hat{\sigma}}
\newcommand \htheta{\hat{\theta}}
\newcommand \s{\sigma}
\newcommand \hy{\hat{y}}
\newcommand \hY{\hat{Y}}
\newcommand \vone{\vec{1}}
\newcommand \e{\varepsilon}
\newcommand \he{\hat{\e}}
\newcommand \z{z}
\newcommand \hVar{\widehat{\Var}}
\newcommand \hCorr{\widehat{\Corr}}
\newcommand \hCov{\widehat{\Cov}}
\newcommand \cN{\mathcal{N}}



\begin{document}

\section*{Project}

\begin{enumerate}

\item Consider the framework of Black and Scholes model.
Let $S_0 =100$, $\sigma = 0.1$, $r=0.05$, $\mu = 0.1$. 

Consider a European type claim that pays you $X_2 = \sqrt{S_2}$ at time moment $T=2$. 

\begin{enumerate}
    \item Simulate 500 trajectories of share price. Plot only 3 of them.
    
    You are free to choose any reasonably small length for time sub-intervals. 

    \item Calculate the arbitrage free price $X_0$ using two approaches:
    theoretical expected value pricing equation and average over 500 simulations. 

    \item How many shares should be in replicating portfolio at time moment $t \in [0;2]$?
    
    Hint: you need to calculate the arbitrage free price $X_t$ and the derivative $\partial X_t/\partial S_t$. 

    \item For each of 500 share price trajectories calculate the number of shares in the replicating portfolio. 
    
    Plot 3 trajectories of the number of shares in the replicating portfolio. 
    
    \item Plot 3 trajectories of the replicating portfolio price. 
    
    \item Plot the scatter plot of 500 final replicating portfolio price versus $X_2$.
    
    Does the portfolio really replicate the claim? Why?
\end{enumerate}

\item Consider a two dimensional stochastic process
\[
    \begin{cases}
    dX_t=1.1X_t dt-Y_t dW_t \\
    dY_t=1.1Y_t dt+X_t dW_t
    \end{cases}
\]
with initial conditions $X_0=1$ and $Y_0=0$.

\begin{enumerate}
    \item Simulate 3 trajectories of this process and plot them on a plane. 
    \item How would you describe the behaviour of this process in plain words?
\end{enumerate}



    
\end{enumerate}


\section*{Notes}

You should answer questions with text and plots and add the code in appendix. 

The report should not exceed \textbf{10 pages}.

You may do the project alone or in small groups of two or three students. 

The report should be upload as one pdf file. 

Deadline - \textbf{2022-04-28, 23:59}. 


\end{document}


