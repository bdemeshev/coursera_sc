% arara: xelatex
%% arara: xelatex

\documentclass[14pt,xcolor=dvipsnames,handout]{beamer}


% !TEX root = om_metrics_14.tex

%\usepackage{epsdice} % dice 1-6 for probability :)

% \usepackage[absolute,overlay]{textpos}

% \usefonttheme[onlymath]{serif}

\usefonttheme{professionalfonts}
% by default beamer changes math fonts for better visibility for projection
% this professionalfonst theme removes this behavior


\usepackage[orientation=portrait,size=custom,width=25.4,height=19.05]{beamerposter}




%25,4 см 19,05 см размеры слайда в powerpoint

\usetheme{metropolis}
\metroset{
  %progressbar=none,
  numbering=none,
  subsectionpage=progressbar,
  block=fill
}

%\usecolortheme{seahorse}

\usepackage{fontspec}
\usepackage{polyglossia}
\setmainlanguage{english}


% \usepackage{fontawesome5} % removed [fixed]
\setmainfont[Ligatures=TeX]{Myriad Pro}
% \setsansfont{Myriad Pro}




% why do we need \newfontfamily:
% http://tex.stackexchange.com/questions/91507/
\newfontfamily{\cyrillicfonttt}{Myriad Pro}
\newfontfamily{\cyrillicfont}{Myriad Pro}
%\newfontfamily{\cyrillicfontbs}{Myriad Pro}
\newfontfamily{\cyrillicfontsf}{Myriad Pro}


% https://tex.stackexchange.com/questions/175860/why-does-unicode-math-break-the-kerning-of-accents-in-combination-with-amssymb
% "You shouldn't be using amssymb together with unicode-math"
\usepackage{amsmath}
\usepackage{amsthm} % amssymb 


% https://tex.stackexchange.com/questions/483722/
% \usepackage[MnSymbol]{mathspec}  % Includes amsmath.
% \usepackage{mathspec}  % Includes amsmath.
% \setmathsfont(Digits,Latin,Greek,Symbols)[Numbers={Lining,Proportional}]{Latin Modern Math}
% mathspec must be loaded earlier than amsmath



%\usepackage{bm}

% \usepackage{fdsymbol} % \nperp

% \usepackage{unicode-math} % \symbf
% \setmathfont{Latin Modern Math}



\usepackage{centernot}

\usepackage{graphicx}

\usepackage{wrapfig}
% \usepackage{animate} % animations :)
% \usepackage{tikz}
%\usetikzlibrary{shapes.geometric,patterns,positioning,matrix,calc,arrows,shapes,fit,decorations,decorations.pathmorphing}
% \usepackage{pifont}
\usepackage{comment}
\usepackage[font=small,labelfont=bf]{caption}
\captionsetup[figure]{labelformat=empty}
% \includecomment{techno}



%Расположение

\setbeamersize{text margin left=15 mm,text margin right=5mm} 
\setlength{\leftmargini}{38 pt}

%\usepackage{showframe}
%\usepackage{enumitem}
% \setlist{leftmargin=5.5mm}


%Цвета от дирекции

\definecolor{dirblack}{RGB}{58, 58, 58}
\definecolor{dirwhite}{RGB}{245, 245, 245}
\definecolor{dirred}{RGB}{149, 55, 53}
\definecolor{dirblue}{RGB}{0, 90, 171}
\definecolor{dirorange}{RGB}{235, 143, 76}
\definecolor{dirlightblue}{RGB}{75, 172, 198}
\definecolor{dirgreen}{RGB}{155, 187, 89}
\definecolor{dircomment}{RGB}{128, 100, 162}

\setbeamercolor{title separator}{bg=dirlightblue!50, fg=dirblue}

%Цвета блоков

% Голубой блок!
\setbeamercolor{block title}{bg=dirblue!30,fg=dirblack}
\setbeamercolor{block title example}{bg=dirlightblue!50,fg=dirblack}
\setbeamercolor{block body example}{bg=dirlightblue!20,fg=dirblack}

\AtBeginEnvironment{exampleblock}{\setbeamercolor{itemize item}{fg=dirblack}}
%\setbeamertemplate{blocks}[rounded][shadow]

% Набор команд для удобства верстки

% Набор команд для структуризации

%\newcommand{\quest}{\faQuestionCircleO}
%\faPencilSquareO \faPuzzlePiece \faQuestionCircleO  \faIcon*[regular]{file} {\textcolor{dirblue}
%\newcommand{\quest}{\textcolor{dirblue}{\boxed{\textbf{?}}}
%\newcommand{\task}{\faIcon{tasks}}
%\newcommand{\exmpl}{\faPuzzlePiece}
%\newcommand{\dfn}{\faIcon{pen-square}}
%\newcommand{\quest}{\textcolor{dirblue}{\faQuestionCircle[regular]}}
%\newcommand{\acc}[1]{\textcolor{dirred}{#1}}
%\newcommand{\accm}[1]{\textcolor{dirred}{#1}}
%\newcommand{\acct}[1]{\textcolor{dirblue}{#1}}
%\newcommand{\acctm}[1]{\textcolor{dirblue}{#1}}
%\newcommand{\accex}[1]{\textcolor{dirblack}{\bf #1}}
%\newcommand{\accexm}[1]{\textcolor{dirblack}{ \mathbf{#1}}}
%\newcommand{\acclp}[1]{\textcolor{dirorange}{\it #1}}
\newcommand{\todo}[1]{\textcolor{dircomment}{\bf #1}}
%\newcommand{\graylink}[1]{{\fontsize{11}{12}\selectfont \textcolor{gray}{#1}}}
%\newcommand{\figcaption}[1]{{\fontsize{18}{20}\selectfont #1}}


\newcommand{\videotitle}[1]{
    {\fontsize{33}{30}\selectfont \textcolor{dirblue}{\textbf{#1}} }

    %\todo{название видеофрагмента}
}

\newcommand{\lecturetitle}[1]{
  {\fontsize{33}{30}\selectfont \textcolor{dirblue}{\textbf{#1}} }

    %\todo{название лекции}
}





%\newcommand{\spcbig}{\vspace{-10 pt}}
%\newcommand{\spcsmall}{\vspace{-5 pt}}

%\usepackage{listings}
%\lstset{
%xleftmargin=0 pt,
%  basicstyle=\small, 
%  language=Python,
  %tabsize = 2,
%  backgroundcolor=\color{mc!20!white}
%}



%\newcommand{\mypart}[1]{\begin{frame}[standout]{\huge #1}\end{frame}}

\setbeamercolor{background canvas}{bg=}

% frame title setup
\setbeamercolor{frametitle}{bg=,fg=dirblue}
\setbeamertemplate{frametitle}[default][left]

\addtobeamertemplate{frametitle}{\hspace*{0.1 cm}}{\vspace*{0.25cm}}


%Шрифты
\setbeamerfont{frametitle}{family=\rmfamily,series=\bfseries,size={\fontsize{33}{30}}}
\setbeamerfont{framesubtitle}{family=\rmfamily,series=\bfseries,size={\fontsize{26}{20}}}


% удобнее знать номер слайда, чтобы вносить правки!  

\setbeamercolor{footline}{fg=dircomment}
\setbeamerfont{footline}{series=\bfseries, size={\fontsize{12}{14}}}
%\setbeamertemplate{footline}[page number]


\defbeamertemplate{footline}{custom footline}
{%
  \hspace*{\fill}%
  \usebeamercolor[fg]{page number in head/foot}%
  \usebeamerfont{page number in head/foot}%
  page: \insertpagenumber\,/\,\insertpresentationendpage%
  \hspace{20pt}%
  slide: \insertframenumber\,/\,\inserttotalframenumber%
  %\hspace*{\fill}
  \vskip2pt%
}
%\setbeamertemplate{footline}[custom footline]

\usepackage{physics}


% tikz block

\usepackage{pgfplots}
\pgfplotsset{compat=newest}

\usepackage{tikz}
\usetikzlibrary{calc}
\usetikzlibrary{quotes,angles}
\usetikzlibrary{arrows}
\usetikzlibrary{arrows.meta}
\usetikzlibrary{positioning,intersections,decorations.markings}
\usetikzlibrary{patterns}
\usepackage{tikzsymbols}

\usepackage{tkz-euclide} 
\usepackage{tikzducks}

%\tikzset{>=latex}


\tikzset{cross/.style={cross out, draw=black, minimum size=2*(#1-\pgflinewidth), inner sep=0pt, outer sep=0pt},
%default radius will be 1pt. 
cross/.default={5pt}}

\colorlet{veca}{red}
\colorlet{vecb}{blue}
\colorlet{vecc}{olive}


\newcommand{\grid}{\draw[color=gray,step=1.0,dotted] (-2.1,-2.1) grid (9.6,6.1)}

% end tikz block

\newcommand{\R}{\mathbb{R}}
\newcommand{\Rot}{\mathrm{R}}
\newcommand{\HH}{\mathrm{H}}
\newcommand{\Id}{\mathrm{I}}
\newcommand{\RR}{\mathbb{R}}
\newcommand{\ZZ}{\mathbb{Z}}
\newcommand{\la}{\lambda}
\let\P\relax
\newcommand{\P}{\mathbb{P}}
\newcommand{\E}{\mathbb{E}}

\newcommand{\cN}{\mathcal{N}}
\newcommand{\cF}{\mathcal{F}}

\newcommand{\dN}{\mathcal{N}}

\newcommand{\qL}{q_{\text{left}}}
\newcommand{\qR}{q_{\text{right}}}



\newcommand{\ba}{\mathbf{a}}
\newcommand{\be}{\mathbf{e}}
\newcommand{\bb}{\mathbf{b}}
\newcommand{\bc}{\mathbf{c}}
\newcommand{\bd}{\mathbf{d}}
\newcommand{\bx}{\mathbf{x}}
\newcommand{\bff}{\mathbf{f}} % \bf is already def
\newcommand{\bv}{\mathbf{v}}
\newcommand{\bzero}{\mathbf{0}}


\DeclareMathOperator{\Var}{Var}
\DeclareMathOperator{\sVar}{sVar}
\DeclareMathOperator{\Cov}{Cov}
\DeclareMathOperator{\sCov}{sCov}
\DeclareMathOperator{\sCorr}{sCorr}
\DeclareMathOperator{\Corr}{Corr}

\DeclareMathOperator{\plim}{plim}


\newcommand{\graylink}[1]{{\fontsize{11}{12}\selectfont \textcolor{gray}{#1}}}
\newcommand{\figcaption}[1]{{\fontsize{18}{20}\selectfont #1}}




\newcommand{\knightduck}{
\raisebox{-3pt}{%
\tikz[scale=0.6]{%
	\duck[helmet]
	\fill[gray!60!black] (0.5831,0.5821) -- (0.8112,0.6468) -- (0.7528,0.7563) .. controls (0.7367,0.7865) and (0.8427,0.8231) .. (0.8497,0.7897) -- (0.8749,0.6705) -- (0.9316,0.5183) -- (0.9744,0.4102) .. controls (0.9865,0.3796) and (0.8848,0.3549) .. (0.8784,0.3872) -- (0.8553,0.5051) -- (0.6311,0.4379) -- (0.6541,0.3774) .. controls (0.6619,0.3569) and (0.6051,0.3422) .. (0.5908,0.3589) .. controls (0.5908,0.3589) and (0.5068,0.4303) .. (0.4875,0.4797) .. controls (0.4692,0.5265) and (0.4818,0.6305) .. (0.4818,0.6305) .. controls (0.4809,0.6553) and (0.5387,0.6862) .. (0.5484,0.6634) -- cycle;
	\fill[gray] (0.8749,0.6705) -- (1.6884,0.9465) -- (1.9041,0.8933) -- (1.7728,0.7282) -- (0.9316,0.5183) -- cycle;%
}}}

\newcommand{\formalduck}{
\raisebox{-3pt}{%
\tikz[scale=0.55]{%
	\duck[tophat,bowtie,tshirt,jacket=gray]%
}}}


\newcommand{\informalduck}{
\raisebox{-3pt}{%
\tikz[scale=0.6]{%
	\duck[crazyhair]%
}}}



\newcommand{\harlequinduck}{
\raisebox{-3pt}{%
\tikz[scale=0.6]{%
    \duck[harlequin=blue,
    niuqelrah=red]%
}}}

\begin{document}


\begin{frame} % название лекции


\lecturetitle{\textbf{Itô's lemma}}

\end{frame}


% !TEX root = ../coursera_sc_03.tex

\begin{frame} % название фрагмента

\videotitle{Itô's lemma}

\end{frame}


\begin{frame}{Itô's lemma: short plan}

  \begin{itemize}[<+->]
    \item Light version for functions of time and Wiener process.
    \item Check the martingale property with Itô's lemma.
    \item More general version.

  \end{itemize}

\end{frame}

\begin{frame}
  \frametitle{Itô's lemma: light version}

  \begin{block}{Informal theorem\informalduck}
    If $Y_t = f(W_t, t)$ then it may be written as 
    \[
      Y_t = Y_0 + \int_0^t f'_W \, dW_u + \int_0^t \left(f'_t + \frac{1}{2} f''_{WW}\right) \, du.
    \]
  \end{block}
  \pause
  \begin{block}{Informal theorem\informalduck}
    If $Y_t = f(W_t, t)$ then it may be written in the short form as 
    \[
      dY_t = f'_W dW_t + f'_t dt + \frac{1}{2}f''_{WW} dt.
    \]
  \end{block}
  
\end{frame}

\begin{frame}
  \frametitle{Exercise}
  Express $Y_t = W_t^3 \cdot t^4$ as an Itô process\knightduck.
  \begin{flalign*}
    \onslide<2->{dY_t & =}\onslide<3->{3W_t^2 t^4\, dW_t + 4W_t^3 t^3 \, dt +}\onslide<4->{ \frac{1}{2}6W_t t^4 dt =& }\\ 
    \onslide<5->{=&3W_t^2 t^4\, dW_t + (4W_t^3 t^3 + 3W_t t^4)\, dt.&}
\end{flalign*} 

\begin{flalign*}
  \onslide<6->{Y_t &= 0 + \int_0^t 3W_u^2 u^4\, dW_u + \int_0^t (4W_u^3 u^3 + 3W_u u^4)\, du. &}
\end{flalign*} 

\end{frame}

\begin{frame}
  \frametitle{Exercise}
  Check whether the process $Y_t = W_t^4 - t^2$ is a martingale\knightduck. 

  \begin{flalign*}
    \onslide<2->{dY_t & =}\onslide<3->{4W_t^3\, dW_t - 2t\, dt +}\onslide<4->{ \frac{1}{2}12W_t^2 dt =& }\\ 
    \onslide<5->{=&4W_t^3 \, dW_t + (6W_t^2 - 2t)\, dt.&}
\end{flalign*} 

\begin{flalign*}
  \onslide<6->{Y_t &= 0 + \int_0^t 4W_u^3 \, dW_u + \int_0^t (6W_u^2 - 2u)\, du. &}
\end{flalign*} 

\onslide<7->{$(Y_t)$ is not a martingale!}
\end{frame}


\begin{frame}
  \frametitle{Exercise}
  Express $Y_t = W_t^2$ as an Itô process and prove the formula for $\int_0^t W_u \, dW_u$\knightduck.

  \begin{flalign*}
    \onslide<2->{dY_t & =}\onslide<3->{2W_t \, dW_t + 0 \, dt +}\onslide<4->{ \frac{1}{2}2 dt =}
    \onslide<5->{2W_t \, dW_t +\, dt.&}
\end{flalign*} 

\begin{flalign*}
  \onslide<6->{Y_t &= 0 + \int_0^t 2W_u, dW_u + \int_0^t 1\, du. & }
\end{flalign*} 

\begin{flalign*}
  \onslide<7->{W_t^2 & = 0 + 2\int_0^t W_u\, dW_u + t. }\onslide<8->{\quad \rightarrow \quad \int_0^t W_u\, dW_u = \frac{1}{2}(W_t^2 - t). & }
\end{flalign*} 

  

\end{frame}
  
\begin{frame}
  \frametitle{Itô's lemma: general version}

  \begin{block}{Informal theorem\informalduck}
    If $Y_t = f(X_t, t)$ where $(X_t)$ is an Itô process then $Y_t$ may be written in the short form as 
    \[
      dY_t = f'_X dX_t + f'_t dt + \frac{1}{2}f''_{XX} (dX_t)^2,
    \]
    where $(dX_t)^2$ is calculated using symbolic rules 
      
    $dt \cdot dW_t = 0$, $dt \cdot dt = 0$, $dW_t\cdot dW_t =dt$.
  \end{block}  
\end{frame}



\begin{frame}
  \frametitle{Exercise}
  Consider $dS_t =\mu S_t dt + \sigma S_t dW_t$ and $Y_t = S_t^2$.

  Find $dY_t$ and recover the full form for $Y_t$\knightduck. 

  \begin{flalign*}
    \onslide<2->{dY_t & =}\onslide<3->{2S_t \, dS_t + 0 \, dt +}\onslide<4->{ \frac{1}{2}2 (dS_t)^2 =&}\\
    \onslide<5->{=2& S_t (\mu S_t dt + \sigma S_t dW_t)  +1 \cdot (\mu S_t dt + \sigma S_t dW_t)^2=&}\\
    \onslide<6->{=2& S_t (\mu S_t dt + \sigma S_t dW_t)  +\sigma^2 S_t^2 dt=&}\\
    \onslide<7->{=&(2\mu S_t^2 + \sigma^2 S_t^2)dt + 2\sigma S_t^2 dW_t.&}
  \end{flalign*} 

\begin{flalign*}
  \onslide<8->{Y_t &= S_0^2 + \int_0^t 2\sigma S_u^2\, dW_u + \int_0^t (2\mu S_u^2 + \sigma^2 S_u^2)\, du. & }
\end{flalign*} 


  

\end{frame}


\begin{frame}
  \frametitle{Itô's lemma: a way to memorize}

  \begin{block}{Informal theorem\informalduck}
    If $Y_t = f(X_t, Z_t, t)$ where $(X_t)$ and $(Z_t)$ are Itô processes then the short form of $Y_t$ may be obtained in two steps: \pause
    \begin{enumerate}
    \onslide<2->{\item Calculate the \alert{second order Taylor expansion} of $f$.}
    \onslide<3->{\item \alert{Simplify} the result using symbolic rules 
      
      $dt \cdot dW_t = 0$, $dt \cdot dt = 0$, $dW_t\cdot dW_t =dt$.}
    \end{enumerate}
  \end{block}
  \onslide<4->{Reminder: $dW_t$, $dY_t$, $dX_t$, $dZ_t$ \alert{do not exist}! 
  
  It's only a \alert{quick way} to find the full form.}
  
\end{frame}




\begin{frame}{Itô's lemma: summary}
  
  \begin{itemize}[<+->]
      \item \alert{Basic tool} to study stochastic integrals.
      \item \alert{Easy to check} whether the process is a martingale.

      \item \alert{Easily written} in short form: 
      
      $dt \cdot dW_t = 0$, $dt \cdot dt = 0$, $dW_t\cdot dW_t =dt$.
  \end{itemize}
    
\end{frame}
  
% !TEX root = ../coursera_sc_02.tex

\begin{frame} % название фрагмента

    \videotitle{Stochastic differential equations}
    
    \end{frame}
    
    
    \begin{frame}{Stochastic differential equations: short plan}
    
      \begin{itemize}[<+->]
        \item Common properties with Riemann integral.
        \item Zero expected value.
        \item Itô isometry.
      \end{itemize}
    
    \end{frame}


\begin{frame}{Stochastic differential equations: summary}
        
\begin{itemize}[<+->]
          \item \alert{Zero} expected value.
          \item Using \alert{Itô isometry} one may calculate variance.
          \item \alert{Some common properties} with Riemann integral.
\end{itemize}
        
\end{frame}
      

% !TEX root = ../coursera_sc_03.tex

\begin{frame} % название фрагмента

    \videotitle{Girsanov theorem}
    
    \end{frame}
    
    
    \begin{frame}{Girsanov theorem: short plan}
    
      \begin{itemize}[<+->]
        \item The \alert{wrong answer} to the pricing problem. 
        \item Idea of \alert{alternative probability}. 
        \item \alert{Girsanov theorem}.
      \end{itemize}
    
    \end{frame}

\begin{frame}
  \frametitle{The wrong answer}

  \begin{block}{Wrong intuition \harlequinduck}
    The future payoff $X_T$ is random, we just need to calculate the expected payoff
    given all available information,
    \[
    X_0 \stackrel{???}{=} \E(X_T \mid \mathcal F_0).
    \]
  \end{block}

  \pause 
  This is true for a martingale, but the claim price is \alert{not a martingale}. 

\end{frame}


\begin{frame}
  \frametitle{Alternative probability}
  \begin{center}
  \begin{tabular}{@{}cccc@{}}
    \toprule
     $A$ & $X=-2$ & $X=0$ & $X=4$  \\ 
     \midrule
     $\P(A)$ & 0.3 & 0.4 & 0.3  \\
     $\P^*(A)$ & 0.4 & 0.1 & 0.5 \\
     \bottomrule
    \end{tabular}
  \end{center}

  \pause
  \[
    \E(X) = 0.6
  \]
  \pause
  \[
    \E^*(X) = 1.2
  \]
\end{frame}

\begin{frame}
  \frametitle{Idea}
  We will introduce a \alert{new probability} $\P^*$ 
  in the Black and Scholes model to simplify the calculation of prices. 
\end{frame}


\begin{frame}{Girsanov theorem}

  \begin{block}{Theorem \formalduck}
    If $(W_t)$ is a Wiener process under probability $\P$ and $W_t^* = b\cdot t + W_t$,
    then there is a probability $\P^*$ such that $(W_t^*)$ is a Wiener process under $\P^*$.
  \end{block}

  \pause
  \[
  \E(W_t) = 0, \pause \quad \E(W_t^*) = b\cdot t, \pause \quad \E^*(W_t^*) = 0.
  \]
        
\end{frame}

\begin{frame}{Girsanov theorem in BS model}

  \begin{block}{Theorem \formalduck}
    In the Black and Scholes model there is an alternative probability $\P^*$ such that
    $(W_t^*)$ is a Wiener process under $\P^*$ and 
    \[
      S_t = S_0 \exp  \left(  \left( \alert{r} - \frac{\sigma^2}{2}\right)t + \sigma W^*_t \right).  
    \]
  \end{block}
  \pause 
  Old formula is still valid,
  \[
    S_t = S_0 \exp  \left(  \left( \alert{\mu} - \frac{\sigma^2}{2}\right)t + \sigma W_t \right), 
  \]
  where $(W_t)$ is a Wiener process under probability $\P$.

\end{frame}

\begin{frame}
  \frametitle{Link between $(W_t)$ and $(W^*_t)$}

  Equivalent formula for share price means that 
  \[
    \left( \alert{\mu} - \frac{\sigma^2}{2}\right)t + \sigma W_t =  \left( \alert{r} - \frac{\sigma^2}{2}\right)t + \sigma W_t^*
  \]
  \pause
  We simplify, 
  \[
    (\mu - r) t  + \sigma W_t = \sigma W_t^* 
  \]
  \pause 
  In \alert{short} form 
  \[
    (\mu - r)dt + \sigma dW_t = \sigma dW_t^*.
  \]

\end{frame}

\begin{frame}
  \frametitle{The meaning of probabilities:}
  \begin{itemize}
    \onslide<1->{\item $\P$ — \alert{real world} probability;}
    \onslide<2->{\item $\P^*$ — \alert{artificial} probability to simplify formulas.}
  \end{itemize}



\end{frame}

\begin{frame}{Girsanov theorem: summary}
        
\begin{itemize}[<+->]
\item Fair price is not a \alert{simple} expected value.
\item \alert{Girsanov theorem} gives equivalent formula for $S_t$:
\begin{flalign*}
  S_t = &S_0  \exp  \left(  \left( \alert{r} - \frac{\sigma^2}{2}\right)t + \sigma W^*_t \right) =& \\
  = & S_0 \exp  \left(  \left( \alert{\mu} - \frac{\sigma^2}{2}\right)t + \sigma W_t \right).&
\end{flalign*}
\item $(W_t^*)$ is a Wiener process under \alert{artificial} probability $\P^*$,
\[
  (\mu - r)dt + \sigma dW_t = \sigma dW_t^*.
\]

\end{itemize}
        
\end{frame}
      



\end{document}
