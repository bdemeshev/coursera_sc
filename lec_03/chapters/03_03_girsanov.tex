% !TEX root = ../coursera_sc_03.tex

\begin{frame} % название фрагмента

    \videotitle{Girsanov theorem}
    
    \end{frame}
    
    
    \begin{frame}{Girsanov theorem: short plan}
    
      \begin{itemize}[<+->]
        \item The \alert{wrong answer} to the pricing problem. 
        \item Idea of \alert{alternative probability}. 
        \item \alert{Girsanov theorem}.
      \end{itemize}
    
    \end{frame}

\begin{frame}
  \frametitle{The wrong answer}

  \begin{block}{Wrong intuition \harlequinduck}
    The future payoff $X_T$ is random, we just need to calculate the expected payoff
    given all available information,
    \[
    X_0 \stackrel{???}{=} \E(X_T \mid \mathcal F_0).
    \]
  \end{block}

  \pause 
  This is true for a martingale, but the claim price is \alert{not a martingale}. 

\end{frame}


\begin{frame}
  \frametitle{Alternative probability}
  \begin{center}
  \begin{tabular}{@{}cccc@{}}
    \toprule
     $A$ & $X=-2$ & $X=0$ & $X=4$  \\ 
     \midrule
     $\P(A)$ & 0.3 & 0.4 & 0.3  \\
     $\P^*(A)$ & 0.4 & 0.1 & 0.5 \\
     \bottomrule
    \end{tabular}
  \end{center}

  \pause
  \[
    \E(X) = 0.6
  \]
  \pause
  \[
    \E^*(X) = 1.2
  \]
\end{frame}

\begin{frame}
  \frametitle{Idea}
  We will introduce a \alert{new probability} $\P^*$ 
  in the Black and Scholes model to simplify the calculation of prices. 
\end{frame}


\begin{frame}{Girsanov theorem}

  \begin{block}{Theorem \formalduck}
    If $(W_t)$ is a Wiener process under probability $\P$ and $W_t^* = b\cdot t + W_t$,
    then there is a probability $\P^*$ such that $(W_t^*)$ is a Wiener process under $\P^*$.
  \end{block}

  \pause
  \[
  \E(W_t) = 0, \pause \quad \E(W_t^*) = b\cdot t, \pause \quad \E^*(W_t^*) = 0.
  \]
        
\end{frame}

\begin{frame}{Girsanov theorem in BS model}

  \begin{block}{Theorem \formalduck}
    In the Black and Scholes model there is an alternative probability $\P^*$ such that
    $(W_t^*)$ is a Wiener process under $\P^*$ and 
    \[
      S_t = S_0 \exp  \left(  \left( \alert{r} - \frac{\sigma^2}{2}\right)t + \sigma W^*_t \right).  
    \]
  \end{block}
  \pause 
  Old formula is still valid,
  \[
    S_t = S_0 \exp  \left(  \left( \alert{\mu} - \frac{\sigma^2}{2}\right)t + \sigma W_t \right), 
  \]
  where $(W_t)$ is a Wiener process under probability $\P$.

\end{frame}

\begin{frame}
  \frametitle{Link between $(W_t)$ and $(W^*_t)$}

  Equivalent formula for share price means that 
  \[
    \left( \alert{\mu} - \frac{\sigma^2}{2}\right)t + \sigma W_t =  \left( \alert{r} - \frac{\sigma^2}{2}\right)t + \sigma W_t^*
  \]
  \pause
  We simplify, 
  \[
    (\mu - r) t  + \sigma W_t = \sigma W_t^* 
  \]
  \pause 
  In \alert{short} form 
  \[
    (\mu - r)dt + \sigma dW_t = \sigma dW_t^*.
  \]

\end{frame}

\begin{frame}
  \frametitle{The meaning of probabilities:}
  \begin{itemize}
    \onslide<1->{\item $\P$ — \alert{real world} probability;}
    \onslide<2->{\item $\P^*$ — \alert{artificial} probability to simplify formulas.}
  \end{itemize}



\end{frame}

\begin{frame}{Girsanov theorem: summary}
        
\begin{itemize}[<+->]
\item Fair price is not a \alert{simple} expected value.
\item \alert{Girsanov theorem} gives equivalent formula for $S_t$:
\begin{flalign*}
  S_t = &S_0  \exp  \left(  \left( \alert{r} - \frac{\sigma^2}{2}\right)t + \sigma W^*_t \right) =& \\
  = & S_0 \exp  \left(  \left( \alert{\mu} - \frac{\sigma^2}{2}\right)t + \sigma W_t \right).&
\end{flalign*}
\item $(W_t^*)$ is a Wiener process under \alert{artificial} probability $\P^*$,
\[
  (\mu - r)dt + \sigma dW_t = \sigma dW_t^*.
\]

\end{itemize}
        
\end{frame}
      
