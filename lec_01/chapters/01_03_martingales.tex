% !TEX root = ../coursera_sc_01.tex


\begin{frame} % название фрагмента

    \videotitle{Martingales}
    
\end{frame}
    
    
\begin{frame}{Martingales: short plan}
    
      \begin{itemize}[<+->]
        \item \alert{Filtration} models the information acquisition.
        \item Definition of a \alert{martingale}.
        \item \alert{Examples} of martingales.
      \end{itemize}
    
\end{frame}

\begin{frame}
    \frametitle{Filtration}

    \pause
    The $\sigma$-algebra $\cF_t$ describes all the information available at time $t$. 

    \pause
    \begin{block}{Definition \formalduck}
        The family of sigma-algebras $(\cF_t, t\geq 0)$ is called \alert{filtration} if it grows in time, 
        $\cF_s \subset \cF_t$ for $s \leq t$.
    \end{block}

    \pause 
    Reminder: Sigma-algebra $\cF_t$ is the collection of events. 

\end{frame}


\begin{frame}
    \frametitle{Natural filtration}

    \pause
    \begin{block}{Definition \formalduck}
        The filtration $(\cF_t, t\geq 0)$ is called a \alert{natural filtration} of a process $(X_t, t\geq 0)$ if
        at time $t$ you have only the information about past values of the process, 
        \[
                \cF_t = \sigma(X_u, u\in [0;t]).  
        \]
    \end{block}

    \pause 
    \begin{block}{Examples}
        Let $(\cF_t)$ be a natural filtration of a Wiener process $(W_t)$. \pause
        
        $\{W_2 < 5\} \in \cF_2$, \pause $\{W_2 > W_5\} \in \cF_6$, \pause 
        
        $\{W_2 < 5\} \not\in \cF_1$, \pause $\{W_2 > W_5 \} \not\in \cF_2$.
    \end{block}

\end{frame}

\begin{frame}
    \frametitle{Martingale}

    \begin{block}{Definition \formalduck}
        Consider a filtration $(\cF_t, t\geq 0)$ and a process $(M_t, t\geq 0)$.

        If the best prediction of the future value $M_t$ of a process is its current value $M_s$ for $s\leq t$,
        \[
          \E(M_t \mid \cF_s) = M_s,  
        \] 
        then $(M_t)$ is called a \alert{martingale} with respect to the filtration $(\cF_t)$.
    \end{block}

    \pause
    Usually we consider natural filtration $(\cF_t)$ of the process $(M_t)$.
\end{frame}


\begin{frame}
    \frametitle{Simple examples}

    \alert{Constant process:} \pause 
    
    If $M_t = 777$ for all $t$ then $\E(M_t \mid \cF_s) = 777 = M_s$.


    \pause
    \vspace*{14pt}


    \alert{Wiener process:} \pause 
    \[
    \E(W_t \mid \cF_s) = \pause  \E(W_s + (W_t - W_s) \mid \cF_s) = \pause W_s + \E(W_t - W_s) = W_s.   
    \]

\end{frame}


\begin{frame}
    \frametitle{More examples}

    \begin{block}{Theorem \formalduck}
        The process $Z_t = W_t^2 - t$ is a \alert{martingale}.
    \end{block}

    \pause
    \begin{block}{Proof \formalduck}
        \begin{flalign*}
            \onslide<2->{\E(W_t^2 &  - t \mid \cF_s) =  }
            \onslide<3->{ \E((W_s + (W_t - W_s))^2  \mid \cF_s) - t = & \\}
            \onslide<4->{ =&\E(W_s^2 + (W_t - W_s)^2 + 2W_s(W_t-W_s) \mid \cF_s) - t =&\\}
            \onslide<5->{ =& W_s^2 + \E((W_t - W_s)^2 \mid \cF_s) + 2W_s \E(W_t-W_s \mid \cF_s) -t  =&\\}
            \onslide<6->{ =&W_s^2 + \E((W_t - W_s)^2) + 2W_s \E(W_t-W_s) -t  =&\\}
            \onslide<7->{ =&W_s^2 + (t-s) + 2W_s \cdot 0 - t = W_s^2  - s&}
        \end{flalign*}
        
    \end{block}

    

\end{frame}


\begin{frame}
    \frametitle{More examples}

    \begin{block}{Theorem \formalduck}
        The process $Z_t = \exp(a W_t - a^2 t/2)$ is a \alert{martingale} for every constant $a$.
    \end{block}

    \pause
    This martingale is very useful in Black and Scholes model. 
\end{frame}




\begin{frame}
    \frametitle{Martingales in discrete time}

    \begin{block}{Theorem \formalduck}
        Consider a filtration $(\cF_t, t\in \{0, 1, 2, \ldots\})$ and a process $(M_t, t \in \{0, 1, 2, \ldots\})$.
        
        \pause
        In discrete time the condition
        \[
            \E(M_t \mid \cF_s) = M_s \text{ for all } s \leq t      
        \]
        \pause
        is completely equivalent to 
        \[
            \E(M_{t+1} \mid \cF_t) = M_t. 
        \]
    \end{block}

\end{frame}

\begin{frame}
    \frametitle{Random walk}
    
    Consider independent and identically distributed  $Z_1$, $Z_2$, \ldots{ } with $\E(Z_t) = 0$.
    \pause
    The cumulative sum
    \[
        S_t = Z_1 + Z_2 + \ldots + Z_t, \text{ with } S_0 = 0
    \]
    is called a \alert{random walk}. 
    \pause

    \begin{block}{Theorem \formalduck}
    The random walk process is a martingale.     
    \pause
    \begin{flalign*}
        \cF_t &=  \sigma(Z_1, Z_2, Z_3, \ldots, Z_t) &       
    \end{flalign*} \pause
    \begin{flalign*}
        \E(S_{t+1} \mid \cF_t) & = \onslide<5->{\E(S_t + Z_{t+1} \mid \cF_t) =} \onslide<6->{S_t + \E(Z_{t+1}) = S_t.} &    
    \end{flalign*}
\end{block}
    
\end{frame}




\begin{frame}{Martingales: summary}

    \begin{itemize}[<+->]
        \item \alert{Filtration} models the information acquisition.
        \item The best prediction of a \alert{martingale} is its current value.
        \item Martingales related to \alert{Wiener process}, \alert{random walk}.
    \end{itemize}
      
\end{frame}
    