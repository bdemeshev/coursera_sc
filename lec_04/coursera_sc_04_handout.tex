% arara: xelatex
%% arara: xelatex


\documentclass[14pt,xcolor=dvipsnames, handout]{beamer}


% !TEX root = om_metrics_14.tex

%\usepackage{epsdice} % dice 1-6 for probability :)

% \usepackage[absolute,overlay]{textpos}

% \usefonttheme[onlymath]{serif}

\usefonttheme{professionalfonts}
% by default beamer changes math fonts for better visibility for projection
% this professionalfonst theme removes this behavior


\usepackage[orientation=portrait,size=custom,width=25.4,height=19.05]{beamerposter}




%25,4 см 19,05 см размеры слайда в powerpoint

\usetheme{metropolis}
\metroset{
  %progressbar=none,
  numbering=none,
  subsectionpage=progressbar,
  block=fill
}

%\usecolortheme{seahorse}

\usepackage{fontspec}
\usepackage{polyglossia}
\setmainlanguage{english}


% \usepackage{fontawesome5} % removed [fixed]
\setmainfont[Ligatures=TeX]{Myriad Pro}
% \setsansfont{Myriad Pro}




% why do we need \newfontfamily:
% http://tex.stackexchange.com/questions/91507/
\newfontfamily{\cyrillicfonttt}{Myriad Pro}
\newfontfamily{\cyrillicfont}{Myriad Pro}
%\newfontfamily{\cyrillicfontbs}{Myriad Pro}
\newfontfamily{\cyrillicfontsf}{Myriad Pro}


% https://tex.stackexchange.com/questions/175860/why-does-unicode-math-break-the-kerning-of-accents-in-combination-with-amssymb
% "You shouldn't be using amssymb together with unicode-math"
\usepackage{amsmath}
\usepackage{amsthm} % amssymb 


% https://tex.stackexchange.com/questions/483722/
% \usepackage[MnSymbol]{mathspec}  % Includes amsmath.
% \usepackage{mathspec}  % Includes amsmath.
% \setmathsfont(Digits,Latin,Greek,Symbols)[Numbers={Lining,Proportional}]{Latin Modern Math}
% mathspec must be loaded earlier than amsmath



%\usepackage{bm}

% \usepackage{fdsymbol} % \nperp

% \usepackage{unicode-math} % \symbf
% \setmathfont{Latin Modern Math}



\usepackage{centernot}

\usepackage{graphicx}

\usepackage{wrapfig}
% \usepackage{animate} % animations :)
% \usepackage{tikz}
%\usetikzlibrary{shapes.geometric,patterns,positioning,matrix,calc,arrows,shapes,fit,decorations,decorations.pathmorphing}
% \usepackage{pifont}
\usepackage{comment}
\usepackage[font=small,labelfont=bf]{caption}
\captionsetup[figure]{labelformat=empty}
% \includecomment{techno}



%Расположение

\setbeamersize{text margin left=15 mm,text margin right=5mm} 
\setlength{\leftmargini}{38 pt}

%\usepackage{showframe}
%\usepackage{enumitem}
% \setlist{leftmargin=5.5mm}


%Цвета от дирекции

\definecolor{dirblack}{RGB}{58, 58, 58}
\definecolor{dirwhite}{RGB}{245, 245, 245}
\definecolor{dirred}{RGB}{149, 55, 53}
\definecolor{dirblue}{RGB}{0, 90, 171}
\definecolor{dirorange}{RGB}{235, 143, 76}
\definecolor{dirlightblue}{RGB}{75, 172, 198}
\definecolor{dirgreen}{RGB}{155, 187, 89}
\definecolor{dircomment}{RGB}{128, 100, 162}

\setbeamercolor{title separator}{bg=dirlightblue!50, fg=dirblue}

%Цвета блоков

% Голубой блок!
\setbeamercolor{block title}{bg=dirblue!30,fg=dirblack}
\setbeamercolor{block title example}{bg=dirlightblue!50,fg=dirblack}
\setbeamercolor{block body example}{bg=dirlightblue!20,fg=dirblack}

\AtBeginEnvironment{exampleblock}{\setbeamercolor{itemize item}{fg=dirblack}}
%\setbeamertemplate{blocks}[rounded][shadow]

% Набор команд для удобства верстки

% Набор команд для структуризации

%\newcommand{\quest}{\faQuestionCircleO}
%\faPencilSquareO \faPuzzlePiece \faQuestionCircleO  \faIcon*[regular]{file} {\textcolor{dirblue}
%\newcommand{\quest}{\textcolor{dirblue}{\boxed{\textbf{?}}}
%\newcommand{\task}{\faIcon{tasks}}
%\newcommand{\exmpl}{\faPuzzlePiece}
%\newcommand{\dfn}{\faIcon{pen-square}}
%\newcommand{\quest}{\textcolor{dirblue}{\faQuestionCircle[regular]}}
%\newcommand{\acc}[1]{\textcolor{dirred}{#1}}
%\newcommand{\accm}[1]{\textcolor{dirred}{#1}}
%\newcommand{\acct}[1]{\textcolor{dirblue}{#1}}
%\newcommand{\acctm}[1]{\textcolor{dirblue}{#1}}
%\newcommand{\accex}[1]{\textcolor{dirblack}{\bf #1}}
%\newcommand{\accexm}[1]{\textcolor{dirblack}{ \mathbf{#1}}}
%\newcommand{\acclp}[1]{\textcolor{dirorange}{\it #1}}
\newcommand{\todo}[1]{\textcolor{dircomment}{\bf #1}}
%\newcommand{\graylink}[1]{{\fontsize{11}{12}\selectfont \textcolor{gray}{#1}}}
%\newcommand{\figcaption}[1]{{\fontsize{18}{20}\selectfont #1}}


\newcommand{\videotitle}[1]{
    {\fontsize{33}{30}\selectfont \textcolor{dirblue}{\textbf{#1}} }

    %\todo{название видеофрагмента}
}

\newcommand{\lecturetitle}[1]{
  {\fontsize{33}{30}\selectfont \textcolor{dirblue}{\textbf{#1}} }

    %\todo{название лекции}
}





%\newcommand{\spcbig}{\vspace{-10 pt}}
%\newcommand{\spcsmall}{\vspace{-5 pt}}

%\usepackage{listings}
%\lstset{
%xleftmargin=0 pt,
%  basicstyle=\small, 
%  language=Python,
  %tabsize = 2,
%  backgroundcolor=\color{mc!20!white}
%}



%\newcommand{\mypart}[1]{\begin{frame}[standout]{\huge #1}\end{frame}}

\setbeamercolor{background canvas}{bg=}

% frame title setup
\setbeamercolor{frametitle}{bg=,fg=dirblue}
\setbeamertemplate{frametitle}[default][left]

\addtobeamertemplate{frametitle}{\hspace*{0.1 cm}}{\vspace*{0.25cm}}


%Шрифты
\setbeamerfont{frametitle}{family=\rmfamily,series=\bfseries,size={\fontsize{33}{30}}}
\setbeamerfont{framesubtitle}{family=\rmfamily,series=\bfseries,size={\fontsize{26}{20}}}


% удобнее знать номер слайда, чтобы вносить правки!  

\setbeamercolor{footline}{fg=dircomment}
\setbeamerfont{footline}{series=\bfseries, size={\fontsize{12}{14}}}
%\setbeamertemplate{footline}[page number]


\defbeamertemplate{footline}{custom footline}
{%
  \hspace*{\fill}%
  \usebeamercolor[fg]{page number in head/foot}%
  \usebeamerfont{page number in head/foot}%
  page: \insertpagenumber\,/\,\insertpresentationendpage%
  \hspace{20pt}%
  slide: \insertframenumber\,/\,\inserttotalframenumber%
  %\hspace*{\fill}
  \vskip2pt%
}
%\setbeamertemplate{footline}[custom footline]

\usepackage{physics}


% tikz block

\usepackage{pgfplots}
\pgfplotsset{compat=newest}

\usepackage{tikz}
\usetikzlibrary{calc}
\usetikzlibrary{quotes,angles}
\usetikzlibrary{arrows}
\usetikzlibrary{arrows.meta}
\usetikzlibrary{positioning,intersections,decorations.markings}
\usetikzlibrary{patterns}
\usepackage{tikzsymbols}

\usepackage{tkz-euclide} 
\usepackage{tikzducks}

\usepackage{booktabs}

%\tikzset{>=latex}


\tikzset{cross/.style={cross out, draw=black, minimum size=2*(#1-\pgflinewidth), inner sep=0pt, outer sep=0pt},
%default radius will be 1pt. 
cross/.default={5pt}}

\colorlet{veca}{red}
\colorlet{vecb}{blue}
\colorlet{vecc}{olive}


\newcommand{\grid}{\draw[color=gray,step=1.0,dotted] (-2.1,-2.1) grid (9.6,6.1)}

% end tikz block

\newcommand{\R}{\mathbb{R}}
\newcommand{\Rot}{\mathrm{R}}
\newcommand{\HH}{\mathrm{H}}
\newcommand{\Id}{\mathrm{I}}
\newcommand{\RR}{\mathbb{R}}
\newcommand{\ZZ}{\mathbb{Z}}
\newcommand{\la}{\lambda}
\let\P\relax
\newcommand{\P}{\mathbb{P}}
\newcommand{\E}{\mathbb{E}}

\newcommand{\cN}{\mathcal{N}}
\newcommand{\cF}{\mathcal{F}}

\newcommand{\dN}{\mathcal{N}}

\newcommand{\qL}{q_{\text{left}}}
\newcommand{\qR}{q_{\text{right}}}



\newcommand{\ba}{\mathbf{a}}
\newcommand{\be}{\mathbf{e}}
\newcommand{\bb}{\mathbf{b}}
\newcommand{\bc}{\mathbf{c}}
\newcommand{\bd}{\mathbf{d}}
\newcommand{\bx}{\mathbf{x}}
\newcommand{\bff}{\mathbf{f}} % \bf is already def
\newcommand{\bv}{\mathbf{v}}
\newcommand{\bzero}{\mathbf{0}}


\DeclareMathOperator{\Var}{Var}
\DeclareMathOperator{\sVar}{sVar}
\DeclareMathOperator{\Cov}{Cov}
\DeclareMathOperator{\sCov}{sCov}
\DeclareMathOperator{\sCorr}{sCorr}
\DeclareMathOperator{\Corr}{Corr}

\DeclareMathOperator{\plim}{plim}


\newcommand{\graylink}[1]{{\fontsize{11}{12}\selectfont \textcolor{gray}{#1}}}
\newcommand{\figcaption}[1]{{\fontsize{18}{20}\selectfont #1}}




\newcommand{\knightduck}{
\raisebox{-10pt}{%
\tikz[scale=0.6]{%
	\duck[helmet]
	\fill[gray!60!black] (0.5831,0.5821) -- (0.8112,0.6468) -- (0.7528,0.7563) .. controls (0.7367,0.7865) and (0.8427,0.8231) .. (0.8497,0.7897) -- (0.8749,0.6705) -- (0.9316,0.5183) -- (0.9744,0.4102) .. controls (0.9865,0.3796) and (0.8848,0.3549) .. (0.8784,0.3872) -- (0.8553,0.5051) -- (0.6311,0.4379) -- (0.6541,0.3774) .. controls (0.6619,0.3569) and (0.6051,0.3422) .. (0.5908,0.3589) .. controls (0.5908,0.3589) and (0.5068,0.4303) .. (0.4875,0.4797) .. controls (0.4692,0.5265) and (0.4818,0.6305) .. (0.4818,0.6305) .. controls (0.4809,0.6553) and (0.5387,0.6862) .. (0.5484,0.6634) -- cycle;
	\fill[gray] (0.8749,0.6705) -- (1.6884,0.9465) -- (1.9041,0.8933) -- (1.7728,0.7282) -- (0.9316,0.5183) -- cycle;%
}}}

\newcommand{\formalduck}{
\raisebox{-10pt}{%
\tikz[scale=0.55]{%
	\duck[tophat,bowtie,tshirt,jacket=gray]%
}}}


\newcommand{\informalduck}{
\raisebox{-10pt}{%
\tikz[scale=0.6]{%
	\duck[crazyhair]%
}}}



\newcommand{\harlequinduck}{
\raisebox{-10pt}{%
\tikz[scale=0.6]{%
    \duck[harlequin=blue,
    niuqelrah=red]%
}}}

\begin{document}


\begin{frame} % название лекции



\lecturetitle{Option pricing}

\end{frame}


% !TEX root = ../coursera_sc_04.tex

\begin{frame} % название фрагмента

\videotitle{Discounted price process}

\end{frame}


\begin{frame}{Discounted price process: plan}

  \begin{itemize}[<+->]
    \item Discounting in \alert{discrete} and \alert{continuous time}. 
    \item Every asset can be \alert{replicated}.
    \item The \alert{pricing formula}. 
  \end{itemize}

\end{frame}

\begin{frame}
  \frametitle{Discounting}

  \begin{block}{Definition in discrete time \formalduck}
    If $X_t$ is the price of a claim at time $t$ and $r$ is the interest rate then  \alert{discounted} price
    is defined as
    \[
    \frac{X_t}{(1 + r)^t} = (1 + r)^{-t} X_t.
    \]
  \end{block}
  \pause
  \begin{block}{Definition in continuous time \formalduck}
    \alert{Discounted} price is defined as
    \[
    \frac{X_t}{(\exp r)^t} =  \frac{X_t}{\exp (rt)} = \exp(-rt) X_t.
    \]
  \end{block}
  \pause 
  For small $r$ the definitions are close as $\exp(r) \approx 1 + r$.

  \pause 
  For $t=0$ discounted price and price are equal. 

\end{frame}


\begin{frame}
  \frametitle{Is the discounted share price a martingale?}

In short form,
\begin{flalign*}
  d(\exp(-rt)S_t)&= -r \exp(-rt) S_t dt + \exp(-rt) dS_t + \frac{0}{2} \cdot (dS_t)^2 =& \\
  =& -r \exp(-rt) S_t dt + \exp(-rt) (\mu S_t dt + \sigma S_t dW_t) = &\\
  =& \exp(-rt) S_t \left( (\mu - r) dt + \sigma dW_t \right).&
\end{flalign*}

\pause
\alert{No}, under $\P$ short form has $dt$ term inside!
\[
S_0 \neq \E(\exp(-rt) S_t \mid \cF_0).
\]

\end{frame}

\begin{frame}
  \frametitle{Is the discounted share price a martingale?}

Let's recall, 
\begin{flalign*}
  d(\exp(-rt)S_t)=\exp(-rt) S_t \left( (\mu - r) dt + \sigma dW_t \right).
\end{flalign*}

\pause
But wait, $(\mu - r) dt + \sigma dW_t = \sigma dW_t^*$, so
\begin{flalign*}
  d(\exp(-rt)S_t)=\exp(-rt) S_t \sigma dW^*_t.
\end{flalign*}

\pause
\alert{Yes}, under $\P^*$ short form has no $dt$ term inside!
\[
S_0 = \E^*(\exp(-rt) S_t \mid \cF_0).
\]
\end{frame}


\begin{frame}
  \frametitle{Replicating strategy}

  \begin{block}{Informal theorem \informalduck}
    In the Black and Scholes model every \alert{european type} asset can be replicated by  
    a \alert{self-financing} stategy that trades shares and risk free bonds. 
    \begin{flalign*}
      &X_t = y_t S_t + z_t B_t,& \\
      &dX_t = y_t dS_t + z_t dB_t. & 
    \end{flalign*}    
  \end{block}
  \pause 
  European type asset gives payoff at a \alert{fixed time moment} $T$.

  \pause
  Self-financing strategy means \alert{no} exogenous capital flow.


\end{frame}

\begin{frame}
  \frametitle{The pricing formula}
  \begin{block}{Informal theorem \informalduck}
    In the Black and Scholes model the discounted price of every \alert{european type} asset
    is a martingale under probability $\P^*$, hence
    \[
    X_0 = \E^*(\exp(-rt)X_t\mid \cF_0) = \exp(-rt) \E^*(X_t \mid \cF_0).  
    \] 
  \end{block}
  \begin{itemize}
    \onslide<2->{\item $(W_t^*)$ is a Wiener process under $\P^*$.}
    \onslide<3->{\item $(\mu - r) dt + \sigma dW_t = \sigma dW_t^*$.}
    \onslide<4->{\item Discounted share price $\exp(-rt)S_t$ is a martingale under $\P^*$.}
  \end{itemize}
\end{frame}

\begin{frame}{Discounted price process: summary}
  
  \begin{itemize}[<+->]
      \item \alert{European claim} gives payoff at a fixed moment of time $T$.
      \item The \alert{discounted price} of any european type claim is 
      a martingale under $\P^*$.
      \item Every European claim may be \alert{replicated}.
      \item The \alert{pricing formula} is
      \[
        X_0 = \E^*(\exp(-rt)X_t\mid \cF_0) = \exp(-rt) \E^*(X_t \mid \cF_0).  
      \] 
  \end{itemize}
    
\end{frame}
  
% !TEX root = ../coursera_sc_04.tex

\begin{frame} % название фрагмента

    \videotitle{Call option price}
    
    \end{frame}
    
    
    \begin{frame}{Call option price: plan}
    
      \begin{itemize}[<+->]
        \item Definition of \alert{call} and \alert{put} options. 
        \item Put-call option \alert{parity}. 
        \item The price of a \alert{call} option. 
      \end{itemize}
    
    \end{frame}
    
    \begin{frame}
      \frametitle{Classic options}
    
      \begin{block}{Definition \formalduck}
        The call option gives a \alert{right} to \alert{buy} one share at a specified strike price $K$ on a specified date $T$.   

        The put option gives a \alert{right} to \alert{sell} one share at a specified strike price $K$ on a specified date $T$.   
      \end{block}
      \pause 
      \[
      C_T = \begin{cases}
          S_T - K, \text{ if } S_T > K; \\
          0, \text{ otherwise.}
      \end{cases}  \quad
      P_T = \begin{cases}
        K - S_T, \text{ if } S_T < K; \\
        0, \text{ otherwise.}        
      \end{cases}  
      \]
    \end{frame}
    
    \begin{frame}
        \frametitle{Put-call parity}
        \[
      C_T = \begin{cases}
          S_T - K, \text{ if } S_T > K; \\
          0, \text{ otherwise.}
      \end{cases}  \quad
      P_T = \begin{cases}
        K - S_T, \text{ if } S_T < K; \\
        0, \text{ otherwise.}        
      \end{cases}  
      \]
    \pause 
    \[
    C_T - P_T = S_T - K    
    \]
    \pause 
    \[
    C_0 - P_0 = S_0 - \exp(-rT) K    
    \]        
    \end{frame}

    \begin{frame}
        \frametitle{Call option price}
        The pricing formula,
        \[
            C_0 = \exp(-rT) \E^*(C_T \mid \cF_0).            
        \]
        \pause
        We rewrite $C_T$ using \alert{indicator} $I = I(S_T > K)$,
        \[
        C_T = I \cdot (S_T - K) = I\cdot S_T - I \cdot K.    
        \]
        \pause 
        Let's split into two terms,
        \begin{flalign*}
            \E^*(C_T \mid \cF_0) &= \E^*(I\cdot S_T - I \cdot K \mid \cF_0) = &\\
         = &\E^*(I\cdot S_T  \mid \cF_0) - \E^*(I \cdot K \mid \cF_0);&
    \end{flalign*}
    \end{frame}


\begin{frame}
    \frametitle{The second term}
    Strike price $K$ is constant,
    \[
        \E^*(I \cdot K \mid \cF_0) = K  \E^*(I\mid \cF_0) = K \P^*(S_T > K \mid \cF_0).
    \]
    \pause
    Let's go down to $W_T^*$,
    \[
    \{S_T > K \} = \{\ln S_t > \ln K \} = \{ \ln S_0 + (r - \sigma^2/2)T + \sigma W_T^* > \ln K\}    
    \]
    \pause
    Or,
    \[
    \{S_T > K \} = \left\{ W_T^* > \frac{\ln K - \ln S_0 - (r - \sigma^2/2)T }{\sigma }  \right\}
    \]
    \pause
    Let's standardise and reverse the inequality,
    \[
        \{S_T > K \} = \left\{ \frac{0 - W_T^*}{\sqrt{T}} < d = \frac{\ln S_0 - \ln K + (r - \sigma^2/2)T }{\sigma\sqrt{T} } \right\}.
    \]
\end{frame}



\begin{frame}
    \frametitle{The second term\ldots}
    We've done one half of the job, 
    \[
        \E^*(I \cdot K \mid \cF_0) = K \P^*(S_T > K \mid \cF_0) = K F(d),
    \]
    where 
    \[
    d = \frac{\ln S_0 - \ln K + (r - \sigma^2/2)T }{\sigma\sqrt{T} }.
    \]

\end{frame}




\begin{frame}
    \frametitle{The final answer}
The first term, 
\begin{flalign*}
    \E^*(I\cdot S_T  \mid \cF_0) & =  \E^*( I(W_T^* < d \sqrt{T}) \cdot S_0 \cdot \exp((r- \sigma^2/2)T + \sigma W_T^*)\mid \cF_0) = &\\
    =& S_0 \exp((r- \sigma^2/2)T) \E^*(I(W_T^* < d \sqrt{T})  \cdot \exp(\sigma W_T^*)) =& \\
    =& S_0 F(d + \sigma \sqrt{T}).&
\end{flalign*}
\pause
The \alert{call option price}, 
\begin{flalign*}
    C_0 &= \exp(-rT) \E^*(C_T \mid \cF_0) = & \\
    &= \exp(-rT)(S_0 F(d + \sigma \sqrt{T}) - K F(d)),&
\end{flalign*}
where $d = \frac{\ln S_0 - \ln K + (r - \sigma^2/2)T }{\sigma\sqrt{T}}$.
    

\end{frame}



    \begin{frame}{Call option price: summary}
      
      \begin{itemize}[<+->]
          \item Call option is the right to \alert{buy} a share, put option is the right to \alert{sell} a share. 
          \item \alert{Put-call} parity between their prices,
          \[
            C_0 - P_0 = S_0 - \exp(-rT) K.    
            \]        
        \item The \alert{call price} is 
        The \alert{call option price}, 
        \begin{flalign*}
            C_0 &= \exp(-rT) \E^*(C_T \mid \cF_0) = & \\
            &= \exp(-rT)(S_0 F(d + \sigma \sqrt{T}) - K F(d)),&
        \end{flalign*}
        where $d = \frac{\ln S_0 - \ln K + (r - \sigma^2/2)T }{\sigma\sqrt{T}}$.
                                      
      \end{itemize}
        
    \end{frame}
      
% !TEX root = ../coursera_sc_04.tex

\begin{frame} % название фрагмента

    \videotitle{Delta hedging}
    
    \end{frame}
    
    
    \begin{frame}{Delta hedging: plan}
    
      \begin{itemize}[<+->]
        \item $dX_t$ using \alert{Itô's lemma}.
        \item $dX_t$ using \alert{replicating strategy}.
        \item The \alert{receipt} for replication.
      \end{itemize}
    
    \end{frame}
    
    \begin{frame}
        \frametitle{Claim price as Itô process}
        The price of a claim $X_t$ is a function of $S_t$ and $t$,
        \[
            X_t = X(S_t, t).
        \]
        \pause
        We need to do some investments to start \alert{replicating} strategy.
        
        \pause 
        Using Itô's lemma, 
        \[
        dX_t = \frac{\partial X}{\partial t} dt + \frac{\partial X}{\partial S} dS_t + \frac{1}{2}\frac{\partial^2 X}{\partial S^2} (dS_t)^2
        \]
        \pause 
        The structure of the answer is
        \[
        dX_t = (\ldots) dt + (\ldots)dW_t.    
        \]
    \end{frame}
    

    \begin{frame}
        \frametitle{Focus on $dW_t$}
        Using Itô's lemma, 
        \[
        dX_t = \frac{\partial X}{\partial t} dt + \frac{\partial X}{\partial S} dS_t + \frac{1}{2}\frac{\partial^2 X}{\partial S^2} (dS_t)^2
        \]
        The structure of the answer is
        \[
        dX_t = (\ldots) dt + (\ldots)dW_t.    
        \]
        \pause
        There are no $dW_t$ in $(dS_t)^2$, only in $dS_t = \mu S_t dt + \sigma S_t dW_t$.
        \pause 
        Hence,
        \[
        dX_t =   (\ldots) dt +  \frac{\partial X}{\partial S} \sigma S_t dW_t.   
        \]
    
    \end{frame}


\begin{frame}
    \frametitle{Replicating portfolio idea}

    We have $y_t$ shares and $z_t$ bonds in portfolio, 
    \[
    X_t = y_t S_t + z_t B_t.    
    \]
    \pause 
    Stochastic integral represents the net cash-flow,
    \[
    X_t = X_0 + \int_0^t y_u \, dS_u + \int_0^t z_u dB_u
    \]
    In short form, 
    \[
    dX_t = y_t dS_t + z_t dB_t.    
    \]
    \pause 
    In Black and Scholes model,
    \[
        dS_t = \mu S_t dt + \sigma S_t dW_t, \quad dB_t = B_t dt.
    \]

\end{frame}


\begin{frame}
    \frametitle{Focus on $dW_t$ again }
    In short form, 
    \[
    dX_t = y_t dS_t + z_t dB_t.    
    \]
    In Black and Scholes model,
    \[
        dS_t = \mu S_t dt + \sigma S_t dW_t, \quad dB_t = B_t dt.
    \]
    \pause
    Hence,
    \[
    dX_t =   (\ldots) dt +  y_t \sigma S_t dW_t.   
    \]
\end{frame}

\begin{frame}
    \frametitle{Delta hedging rule}
    \begin{block}{Informal theoerm \informalduck}
        To replicate a european type claim with price $X(S_t, t)$ we should hold $y_t$ shares and $z_t$ bonds, 
        where 
        \[
            \begin{cases}
                y_t = \frac{\partial X}{\partial S}; \\
                z_t = \frac{X_t - y_t S_t}{B_t}.    
            \end{cases}
        \]
    \end{block}
\end{frame}



    \begin{frame}{Delta hedging: summary}
      
      \begin{itemize}[<+->]
          \item From \alert{Itô's lemma}
          \[
          dX_t = \ldots \cdot  dt + \frac{\partial X}{\partial S} \sigma S_t \, dW_t.    
          \]

          \item From \alert{self-financing} assumptions
          \[
          dX_t = \ldots \cdot  dt + y_t \sigma S_t \, dW_t,    
          \]
          where $y_t$  is the amount of shares we hold at $t$.

          \item The \alert{delta-hedging} rule
          \[
          y_t =  \frac{\partial X}{\partial S}.
          \]

      \end{itemize}
      \pause
      Thank you! \knightduck \formalduck \informalduck \harlequinduck

    \end{frame}
      

\end{document}
